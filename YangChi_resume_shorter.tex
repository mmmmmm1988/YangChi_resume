\documentclass[line]{res}

\addtolength{\oddsidemargin}{-.5in}
\addtolength{\evensidemargin}{-.5in}
\addtolength{\textwidth}{.75in}

\addtolength{\topmargin}{-.5in}

\setlength{\textheight}{11.5in} % increase text height to fit on 1-page 

\usepackage{hyperref}

\hyphenation{Systems RESTful}

\begin{document}
\name{Yang Chi}
\address{350 Elan Village Ln \#111 \\ San Jose, CA 95134 \\ 513-679-0772}
\address{Email: \href{mailto:yang@yangchi.me}{yang@yangchi.me} \\ Github: \url{https://github.com/yangchi} }

\begin{resume}
	\section{EXPERIENCE}
	\begin{description}
		\item[12/2013 - Present:] Software Engineer, Cisco Systems \\
		I work in the Core Software Group (CSG) at Cisco. Specifically I work on network policy infra across many platforms on Cisco IOS systems. Some interesting works that I contributed:
        \begin{itemize}
            \item Code convergence of the policy infra between two major branches of Cisco IOS systems.
            \item WPM (Wireless Provisioning Module): A event-driven module sits in-between wireless controller and policy infra to help applying dynamic policies onto wireless entities.
			\item Non-unique class ID in the policy infra to support non-unique priority levels in OpenFlow.
		\end{itemize}
	\end{description}
	\section{EDUCATION}
	\begin{description}
		\item[09/2008 - 12/2013:] Ph.D. in Computer Science \& Engineering, University of Cincinnati\\
			Dissertation: Effective Use of Network Coding in Multi-hop Wireless Networks
		\item[09/2004 - 04/2008:] B.S. in Network Engineering (CS equivalent), Chongqing University
	\end{description}
	
	\section{SKILLS}
	\begin{itemize}
		\item Programming languages: C \textgreater \ C++ = Python \textgreater \ Java \textgreater \ Go = PHP \textgreater \ JavaScript
		\item Non-linguistics: SQL, HTML, CSS
		\item Experience: Linux/Unix programming, Network Simulation (ns-3), OpenFlow and SDN, RESTful API, Web Development
		\item Strong background in Computer Networks and TCP/IP Protocols and solid knowledge in Operating Systems, Distributed Computing, Algorithms and Data Structure. Fundamental knowledge in databases (SQL and NoSQL)
	\end{itemize}

	\section{RESEARCH WORKS}
	\begin{description}
		\item[Network Coding in Multi-Radio Networks:]
			Design and implement an opportunistic and independent 2.5 layer protocol for network coding in multi-radio networks. First distributed and practical solution to this problem. Implemented in ns-3 with C++. Throughput gain in some cases can be 10\%. Latency gain around 50\% is also achieved. \\
			Publication: \emph{Murco: An Opportunistic Network Coding Framework in Multi-Radio Networks, IEEE ICC 2012, first author}
		\item[Practical Coding-Aware Routing Protocols:]
			Propose and design a new routing metric ETOX and a hybrid routing protocol HyCare for network coding capable networks. Implemented in ns-3 with C++. Achieve around 100\% throughput gain compared to classical routing protocols with network coding in wireless mesh network backbone.\\
			Publication: \emph{HyCare: Hybrid Coding-Aware Routing with ETOX Metric in Multi-hop Wireless Networks, to appear at IEEE MASS 2013, first author}
		\item[Network Locality in Wireless Networks:]
			This is the first validation of network locality in both WLAN and mesh networks. Created packet parser in Python to analyze more than 1.3 billion packets (more than 130GB of data) collected from both real network traces and simulations. \\
			Publication: \emph{Network Locality in Wireless Networks, ACS/IEEE AICCSA 2013, first author}
		\item [Decoding-Delay Sensitive Coding Scheme in TCP:]
			Designing a novel coding scheme for network-coded TCP to solve the decoding delay problem in such TCP implementations. \\
			\textbf{Paper: \href{http://arxiv.org/abs/1408.2626}{TCP-Forward: Fast and Reliable TCP Variant for Wireless Networks \\ (http://arxiv.org/abs/1408.2626)}}
	\end{description}
	
	\section{OTHER EXPERIENCE}
	\begin{description}
		\item[Real-time Server] Personal project. I wrote the back-end for my friend's online game with Python, Tornado and Redis.
		\item[Web Developer] University of Cincinnati, 04/2012 - 06/2013: Implement E-Portfolio system, an online portfolio and assessment platform, for College of Engineering and Applied Sciences at University of Cincinnati using PHP and MySQL. 
	\end{description}

\end{resume}
\end{document}
