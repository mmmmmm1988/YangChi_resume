\documentclass[line]{res}
\usepackage{hyperref}

%\addtolength{\oddsidemargin}{-.5in}
%\addtolength{\evensidemargin}{-.5in}
%\addtolength{\textwidth}{.75in}

%\addtolength{\topmargin}{-.5in}

%\setlength{\textheight}{9.5in} 

\usepackage{hyperref}

\hyphenation{Systems RESTful}

\begin{document}
\name{Yang Chi}
\address{36681 Bishop St\\ Newark, CA 94560 \\ 513-679-0772}
\address{Email: \href{mailto:yang@yangchi.me}{yang@yangchi.me} \\ Github: \url{https://github.com/yangchi} }

\begin{resume}
	\section{EXPERIENCE}
	\begin{description}
		\item[12/2013 - Present:] Software Engineer, Cisco Systems \\
		I work in the Core Software Group (CSG) at Cisco. Specifically I work on network policy infra across many platforms on Cisco IOS systems. Some interesting works that I contributed:
            \begin{itemize}
            \item Code convergence of the network policy infra between two major branches of Cisco IOS systems.
			\item Remote Class lookup API: provide an API for remote QoS class lookup in Cisco IOS.
            \item WPM (Wireless Provisioning Module): A event-driven module sits in-between wireless controller and policy infra to help applying dynamic policies onto wireless entities like AP, radio and clients.
            \item Non-unique class ID in the policy infra to support non-unique priority levels in OpenFlow.
            \item Network policy statistics: This is a small one. Just used an AVL tree to count and update policy types and client types, for the speed.
            \end{itemize}
	\end{description}
	\section{EDUCATION}
	\begin{description}
		\item[09/2008 - 12/2013:] Ph.D. in Computer Science \& Engineering, University of Cincinnati, \\
			GPA: 3.8, \emph{Dissertation: Effective Use of Network Coding in Multi-hop Wireless Networks}
		\item[09/2004 - 04/2008:] B.S. in Network Engineering (CS equivalent), Chongqing University
	\end{description}
	
	\section{SKILLS}
	\begin{itemize}
		\item Programming Languages: C \textgreater \ C++ = Python \textgreater \ Java \textgreater \ Go = PHP \textgreater \ JavaScript
		\item Non-linguistics: SQL, HTML, CSS
		\item Experience: Linux/Unix programming, Network Simulation (ns-3), OpenFlow and SDN (Mininet, POX), RESTful API, Web Development
		\item Strong background in Computer Networks and TCP/IP Protocol Stack.
		\item Solid knowledge in Operating Systems, Distributed Computing, Algorithms and Data Structure. Fundamental knowledge in databases (SQL and NoSQL)

	\end{itemize}

	\section{PAST RESEARCH WORKS}
	\begin{description}
		\item[Network Coding in Multi-Radio Networks:] 
			Design and implement an opportunistic and independent 2.5 layer protocol for network coding in multi-radio networks. First distributed and practical solution to this problem. Throughput gain in some cases can be 10\%. Latency gain around 50\% is also achieved. \\
			\textbf{Source code of Yanci, in C++, my implementation of COPE: \url{https://github.com/yangchi/ns3-yanci}} \\
			\textbf{Source code Murco in C++: \url{https://github.com/yangchi/Murco}} \\
			Publication: \emph{Murco: An Opportunistic Network Coding Framework in Multi-Radio Networks, IEEE ICC 2012, first author}
		\item[Practical Coding-Aware Routing Protocols:] 
			Propose and design a new routing metric ETOX and a hybrid routing protocol HyCare for network coding capable networks. ETOX consider both coding opportunities and wireless channel quality. HyCare has both link-state routing and reverse forwarding functions. Achieve around 100\% throughput gain compared to classical routing protocols with network coding in wireless mesh network backbone.\\
			\textbf{Source code in C++: \url{https://github.com/yangchi/ns3-ETOX} } \\
			Publication: \emph{HyCare: Hybrid Coding-Aware Routing with ETOX Metric in Multi-hop Wireless Networks, to appear at IEEE MASS 2013, first author}
		\item[Network Locality in Wireless Networks:] 
			This is the first validation of network locality in both WLAN and wireless mesh networks. Created packet parser in \textbf{Python} to analyze more than 1.3 billion packets (more than 130GB of data) collected from both real network traces and simulations. Examined 4 common network locality characteristics with 5 different routing schemes in multi-hop wireless networks.\\
			\textbf{Source code in Python: \url{https://bitbucket.org/yangchi/trace_parser} } \\
			Publication: \emph{Network Locality in Wireless Networks, ACS/IEEE AICCSA 2013, first author}
		\item [Decoding-Delay Sensitive Coding Scheme in TCP:]
			Design a novel coding scheme for network-coded TCP to solve the decoding delay problem in such TCP implementations. Constant first-packet decoding delay in regardless of the number of raw packets in an encoded packet. 10\% throughput increase and even more significant end-to-end delay reduce compared to previous similar algorithm. \\
			Paper: \href{http://arxiv.org/abs/1408.2626}{\emph{TCP-Forward: Fast and Reliable TCP Variant for Wireless Networks}  \\ (http://arxiv.org/abs/1408.2626)}
	\end{description}
	
	\section{OTHER WORKS}
	\begin{description}
		\item[Real-time Server] Personal project for fun. My friend was writing an online game. I wrote the back-end for him with Python, Tornado and Redis.
		\item[Web Developer] University of Cincinnati, 04/2012 - 06/2013: Design and implement E-Portfolio system, an online portfolio and assessment platform, for College of Engineering and Applied Sciences at University of Cincinnati using \textbf{PHP} and \textbf{MySQL} as well as \textbf{HTML, CSS and JavaScript}. \\
		\textbf{Source code, mostly in PHP: \url{https://bitbucket.org/yangchi/e-portfolio} }
		\item[Teaching Assistant] University of Cincinnati, Spring 2010-2011: Ad Hoc and Sensor Networks class.
		\item[Internship] Institute of Computing Technology, Chinese Academy of Sciences, Summer 2007: Tested functions and reliability of a SOA framework with \textbf{Java}.
	\end{description}

\end{resume}
\end{document}
