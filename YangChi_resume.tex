\documentclass[line]{res}
\setlength{\textheight}{9.5in} % increase text height to fit on 1-page 

\usepackage{hyperref}

\begin{document}
\name{Yang Chi}
\address{226 Ludlow Ave Apt 11 \\ Cincinnati, OH 45220 \\ 513-679-0772}
\address{Email: \href{mailto:yang@yangchi.me}{yang@yangchi.me} \\ Github: \url{https://github.com/yangchi} }

\begin{resume}
	\section{EDUCATION}
	\begin{description}
		\item[09/2008 - Present:] Ph.D. in Computer Science \& Engineering, University of Cincinnati\\
			Dissertation: Effective Use of Network Coding in Multi-hop Wireless Networks
		\item[09/2004 - 04/2008:] B.S. in Network Engineering (CS equivalent), Chongqing University
	\end{description}
	
	\section{SKILLS}
	\begin{itemize}
		\item Everyday Languages: C, C++, Python, PHP, LaTeX.
		\item Also know: HTML, CSS, MySQL, Shell, JavaScript, Java, JSP, Ruby.
		\item Experience: Linux/Unix programming environment, protocol stack in Linux kernel, Network Simulation (ns-3), Virtualization, Cloud service (OpenShift, Heroku), OpenFlow and SDN (Mininet, POX, Floodlight)
		\item Strong background in Computer Networks and TCP/IP Protocol Stack.
		\item Solid knowledge in Operating Systems, Distributed Computing, Algorithms and Data Structure.
	\end{itemize}

	\section{RESEARCH WORKS}
	\begin{description}
		\item[Network Coding in Multi-Radio Networks:]
			Design and implement an opportunistic and independent 2.5 layer protocol for network coding in multi-radio networks. First distributed and practical solution to this problem. Fully implemented in ns-3 with C++. Throughput gain in some cases can be 10\%. Latency gain around 50\% is also achieved. \\
			Publication: \emph{Murco: An Opportunistic Network Coding Framework in Multi-Radio Networks, IEEE ICC 2012, first author}
		\item[Practical Coding-Aware Routing Protocols:]
			Propose and design a new routing metric ETOX and a hybrid routing protocol HyCare for network coding capable networks. ETOX consider both coding opportunities and wireless channel quality. HyCare has both link-state routing and reverse forwarding functions. Fully implemented in ns-3 with C++. Achieve around 100\% throughput gain compared to classical routing protocols with network coding in wireless mesh network backbone.\\
			Publication: \emph{HyCare: Hybrid Coding-Aware Routing with ETOX Metric in Multi-hop Wireless Networks, submitted to IEEE MASS 2013, first author}
		\item[Network Locality in Wireless Networks:]
			This is the first validation of network locality in both WLAN and wireless mesh networks. Created packet parser in Python to analyze more than 1.3 billion packets (more than 130GB of data) collected from both real network traces and simulations. Examined 4 common network locality characteristics with 5 different routing schemes in multi-hop wireless networks.\\
			Publication: \emph{Network Locality in Wireless Networks, to appear at ACS/IEEE AICCSA 2013, first author}
		\item [Decoding-Delay Sensitive Coding Scheme in TCP:]
			Designing a novel coding scheme for network-coded TCP to solve the decoding delay problem in such TCP implementations.
	\end{description}
	
	\section{OTHER EXPERIENCE}
	\begin{description}
		\item[Web Developer] University of Cincinnati, 04/2012 - 06/2013: Implement E-Portfolio system, an online portfolio and assessment platform, for College of Engineering and Applied Sciences at University of Cincinnati using PHP and MySQL. 
		\item[Teaching Assistant] University of Cincinnati, Spring 2010-2011: in Ad Hoc and Sensor Networks class.
		\item[Internship] Institute of Computing Technology, Chinese Academy of Sciences, Summer 2007: Tested functions and reliability of a SOA framework with Java.
	\end{description}

\end{resume}
\end{document}
